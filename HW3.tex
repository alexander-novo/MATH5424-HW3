\documentclass{article}

\usepackage[margin=.75in]{geometry}
\usepackage{amsthm,mathtools}
\usepackage[USenglish]{babel}
\usepackage{hyperref, xurl}     % Used for better URL formatting
\usepackage{microtype} % Help text fit on the page more, rather than spilling into the margins
\usepackage{fancyhdr}
\usepackage[inline]{enumitem}
\usepackage{cleveref}

\usepackage{physics}
\usepackage{pgfplots}
\usepackage{cancel}
\usepackage{halloweenmath}
\usepackage[newfloat]{minted}
\usepackage{xcolor, mdframed}
\usepackage{caption}
\usepackage{subcaption}
\usepackage{tasks}
\usepackage[linesnumbered,ruled]{algorithm2e}
\usepackage{unicode-math}
\usepackage{float}

\setmathfont{latinmodern-math.otf}
\setmathfont{texgyrepagella-math.otf}[range=bb]

\pgfplotsset{compat=1.18}
\usepgfplotslibrary{fillbetween}
\usepgfplotslibrary{groupplots}

% Clickable link configuration
\hypersetup{
	linktoc = all,      % Have the entire TOC line link to the page
	pdfborder = {0 0 0} % Remove link borders
}

\crefname{enumi}{part}{parts}
\crefname{algocf}{algorithm}{algorithms}

\theoremstyle{definition}
\newtheorem*{answer}{Answer}

\definecolor{lightgray}{gray}{0.95}
\newmintedfile[rustcode]{rust}{
	linenos,
	firstnumber=1,
	tabsize=2,
	% bgcolor=lightgray,
	% frame=single,
	breaklines,
	% texcomments % Turned off due to the presence of _ characters in many comments
	escapeinside = \$\$,
}

\newenvironment{code}{\captionsetup{type=listing}}{}
\SetupFloatingEnvironment{listing}{name=Output}

% Title page information
\title{Homework 3\\{\large Math 5424}\\{\large Numerical Linear Algebra}}
\author{Alexander Novotny\\Zuriah Quinton}

% Setup headers for page numbers
\pagestyle{fancy}
\rhead{Novotny \& Quinton}
\lhead{Math 5424 HW 3}

\pagenumbering{arabic} % Change page numbering to lowercase roman for pre-pages

\renewcommand{\qedsymbol}{$\bigpumpkin$}

\newcommand{\comp}  {\mathbb{C}}
\newcommand{\reals}{\mathbb{R}}
\newcommand{\ints}  {\mathbb{Z}}
\newcommand{\mat}[1]{\symbfit{#1}}
\newcommand{\fexp}[1]{\phantom{^{#1}}2^{#1}}
\DeclareMathOperator{\sgn}{sgn}
\DeclareMathOperator{\cond}{cond}
\DeclareMathOperator{\fl}{fl}
\DeclareMathOperator*{\argmax}{arg\,max}
\DeclareMathOperator{\range}{Range}
\DeclareMathOperator{\nullsp}{Null}

\newtheorem{lemma}{Lemma}
\DeclareMathOperator{\spn}{span}

\begin{document}

\maketitle

\begin{enumerate}[leftmargin=\labelsep]
	% 1
	\item Let A be \(n \times m\) with \(n > m\). Show that \(\|\mat{A}^\top\mat{A}\|_2 = \|\mat{A}\|_2^2\) and \(\kappa_2 (\mat{A}^\top\mat{A}) = \kappa_2 (\mat{A})^2\).

	      Let \(M\) be \(n \times n\) and positive definite and L be its Cholesky factor so that \(M = \mat{L}\mat{L}^\top\). Show that \(\|M\|_2 = \|L\|_2^2\) and \(\kappa_2(\mat{M}) = \kappa_2(\mat{L})^2\).
	      \begin{proof}
		      We have
		      \begin{equation*}
			      \norm{\mat{A}^\top \mat{A}}_2 = \max_i \abs{ \lambda_i(\mat{A}^\top \mat{A}) } %\label{eq:2normnormal}
		      \end{equation*}
		      since \(\mat{A}^\top \mat{A}\) is normal, and
		      \begin{equation*}
			      \norm{\mat{A}}^2_2 = \sqrt{ \max_i \lambda_i(\mat{A}^\top \mat{A}) } ^2= \max_i \abs{ \lambda_i(\mat{A}^\top \mat{A}) }.
		      \end{equation*}
		      So \(\|\mat{A}^\top\mat{A}\|_2 = \|\mat{A}\|_2^2\).
		      We have
		      \begin{equation*}
			      \norm{\mat{A}^\top \mat{A}}_2 = \max_i \abs{ \lambda_i(\mat{A}^\top \mat{A}) } %\label{eq:2normnormal}
		      \end{equation*}
		      since \(\mat{A}^\top \mat{A}\) is normal, and
		      \begin{equation*}
			      \norm{\mat{A}}^2_2 = \sqrt{ \max_i \lambda_i(\mat{A}^\top \mat{A}) } ^2= \max_i \abs{ \lambda_i(\mat{A}^\top \mat{A}) }.
		      \end{equation*}
		      So \(\|\mat{A}^\top\mat{A}\|_2 = \|\mat{A}\|_2^2\).

		      Next, we have
		      \begin{align*}
			      \kappa_2(\mat{A}^\top \mat{A}) & = \norm{(\mat{A}^\top \mat{A})^{-1}}_2 \norm{\mat{A}^\top \mat{A}}_2                                   \\
			                                     & = \max_i \abs{ \lambda_i((\mat{A}^\top \mat{A})^{-1} )} \max_i \abs{ \lambda_i(\mat{A}^\top \mat{A}) } \\
			                                     & = \frac{1}{\sigma_m^2} \sigma_1^2                                                                      \\
			                                     & = \frac{\sigma_1^2}{\sigma_m^2},
		      \end{align*}
		      where \(\sigma_i\) is the \(i^{\text{th}}\) singular value of \(\mat{A}\) in descending order. Further,
		      \begin{align*}
			      \kappa_2(\mat{A})^2 & = \qty(\frac{\sigma_1}{\sigma_m})^2 \\
			                          & = \frac{\sigma_1^2}{\sigma_m^2}.
		      \end{align*}
		      Note, we may have \(\sigma_m=0\) if \(\mat{A}\) is not full rank. In this case, we still have \(\kappa_2(\mat{A}^\top \mat{A})=\infty=\kappa_2(\mat{A})^2\).

		      Next, for \(\mat{M} = \mat{L}\mat{L}^\top\), we have
		      \begin{align*}
			      \norm{\mat{M}}_2              & = \norm{\mat{L}\mat{L}^\top}_2 = \max_i \abs{ \lambda_i(\mat{L}\mat{L}^\top) }                   \\
			      \text{and }\norm{\mat{L}}_2^2 & = \sqrt{\max_i\lambda_i(\mat{L}^\top\mat{L})}^2 = \max_i \abs{ \lambda_i(\mat{L}^\top\mat{L}) }.
		      \end{align*}
		      By \cref{lem:eigenvalsLLT} these are equivalent.
		      Next, for \(\mat{M} = \mat{L}\mat{L}^\top\), we have
		      \begin{align*}
			      \norm{\mat{M}}_2              & = \norm{\mat{L}\mat{L}^\top}_2 = \max_i \abs{ \lambda_i(\mat{L}\mat{L}^\top) }                   \\
			      \text{and }\norm{\mat{L}}_2^2 & = \sqrt{\max_i\lambda_i(\mat{L}^\top\mat{L})}^2 = \max_i \abs{ \lambda_i(\mat{L}^\top\mat{L}) }.
		      \end{align*}
		      By \cref{lem:eigenvalsLLT} these are equivalent.

		      Finally, for the condition number of \(\mat{M}\) we have
		      \begin{align*}
			      \kappa_2(\mat{M}) & = \kappa_2(\mat{L}\mat{L}^\top)                                                                          \\
			                        & = \norm{(\mat{L}\mat{L}^\top)^{-1}}_2 \norm{\mat{L}\mat{L}^\top}_2                                       \\
			                        & = \max_i \abs{ \lambda_i\qty((\mat{L}\mat{L}^\top)^{-1} )} \max_i \abs{ \lambda_i(\mat{L}\mat{L}^\top) } \\
			                        & = \frac{1}{\lambda_m} \lambda_1                                                                          \\
			                        & = \frac{\sigma_1^2}{\sigma_m^2}.
		      \end{align*}
		      Since \(\mat{M}\) is positive definite, we know \(\sigma_i>0\) for all \(i\), so this value is finite. Further,
		      \begin{align*}
			      \kappa_2(\mat{L})^2 & = \qty(\frac{\sigma_1}{\sigma_m})^2 \\
			                          & = \frac{\sigma_1^2}{\sigma_m^2}.
		      \end{align*}
		      Thus, \(\kappa_2(\mat{M}) = \kappa_2(\mat{L})^2\).
		      Finally, for the condition number of \(\mat{M}\) we have
		      \begin{align*}
			      \kappa_2(\mat{M}) & = \kappa_2(\mat{L}\mat{L}^\top)                                                                          \\
			                        & = \norm{(\mat{L}\mat{L}^\top)^{-1}}_2 \norm{\mat{L}\mat{L}^\top}_2                                       \\
			                        & = \max_i \abs{ \lambda_i\qty((\mat{L}\mat{L}^\top)^{-1} )} \max_i \abs{ \lambda_i(\mat{L}\mat{L}^\top) } \\
			                        & = \frac{1}{\lambda_m} \lambda_1                                                                          \\
			                        & = \frac{\sigma_1^2}{\sigma_m^2}.
		      \end{align*}
		      Since \(\mat{M}\) is positive definite, we know \(\sigma_i>0\) for all \(i\), so this value is finite. Further,
		      \begin{align*}
			      \kappa_2(\mat{L})^2 & = \qty(\frac{\sigma_1}{\sigma_m})^2 \\
			                          & = \frac{\sigma_1^2}{\sigma_m^2}.
		      \end{align*}
		      Thus, \(\kappa_2(\mat{M}) = \kappa_2(\mat{L})^2\).
	      \end{proof}

	      % lemma for 1
	      \begin{lemma}\label{lem:eigenvalsLLT}
		      If \(\vec{v}\) is an eigenvector of \(\mat{L}\mat{L}^\top\), with \(\lambda\) its eigenvalue, then \(\lambda\) is an eigenvalue of \(\mat{L}^\top\mat{L}\), with eigenvector \(\mat{L}^\top \vec{v}\).
	      \end{lemma}
	      \begin{proof}
		      We have
		      \begin{alignat}{4}
			               &  & \mat{L}\mat{L}^\top \vec{v}                          & = \lambda \vec{v} \label{eq:eigenval}         \\
			      \implies &  & \qty(\mat{L}^\top\mat{L}) \qty(\mat{L}^\top \vec{v}) & = \lambda \qty(\mat{L}^\top\vec{v}).\nonumber
		      \end{alignat}
		      For \(\vec{v}\not\in\ker(\mat{L}^\top)\). Suppose for contradiction that \(\vec{v}\in\ker(\mat{L}^\top)\). Then \cref{eq:eigenval} is
		      \[
			      \mat{L}\underbrace{\mat{L}^\top \vec{v}}_{=0} = \lambda \vec{v} \implies \lambda = 0,
		      \]
		      which is absurd since the eigenvalues of a positive definite matrix are non-zero. Thus, we have \(\vec{v}\not\in\ker(\mat{L}^\top)\), and so \(\lambda\) is always an eigenvalue of \(\mat{L}^\top\mat{L}\).
	      \end{proof}
	      % lemma for 1
	      \begin{lemma}\label{lem:eigenvalsLLT}
		      If \(\vec{v}\) is an eigenvector of \(\mat{L}\mat{L}^\top\), with \(\lambda\) its eigenvalue, then \(\lambda\) is an eigenvalue of \(\mat{L}^\top\mat{L}\), with eigenvector \(\mat{L}^\top \vec{v}\).
	      \end{lemma}
	      \begin{proof}
		      We have
		      \begin{alignat}{4}
			               &  & \mat{L}\mat{L}^\top \vec{v}                          & = \lambda \vec{v} \label{eq:eigenval}         \\
			      \implies &  & \qty(\mat{L}^\top\mat{L}) \qty(\mat{L}^\top \vec{v}) & = \lambda \qty(\mat{L}^\top\vec{v}).\nonumber
		      \end{alignat}
		      For \(\vec{v}\not\in\ker(\mat{L}^\top)\). Suppose for contradiction that \(\vec{v}\in\ker(\mat{L}^\top)\). Then \cref{eq:eigenval} is
		      \[
			      \mat{L}\underbrace{\mat{L}^\top \vec{v}}_{=0} = \lambda \vec{v} \implies \lambda = 0,
		      \]
		      which is absurd since the eigenvalues of a positive definite matrix are non-zero. Thus, we have \(\vec{v}\not\in\ker(\mat{L}^\top)\), and so \(\lambda\) is always an eigenvalue of \(\mat{L}^\top\mat{L}\).
	      \end{proof}

	      % 2
	\item In this question we will ask how to solve \(\mat{B}\vec{y} = \vec{c}\) given a fast way to solve \(\mat{A}\vec{x} = \vec{b}\), where \(\mat{A} - \mat{B}\) is ``small'' in some sense.
	      \begin{enumerate}
		      \item Prove the \emph{Sherman-Morrison formula}: Let \(\mat{A}\) be nonsingular, \(\vec{u}\) and \(\vec{v}\) be column vectors, and \(\mat{A} + \vec{u}\vec{v}^\top\) be nonsingular. Then \((\mat{A} + \vec{u}\vec{v}^\top)^{-1} = \mat{A}^{-1} - (\mat{A}^{-1}\vec{u}\vec{v}^\top\mat{A}^{-1})/(1 + \vec{v}^\top \mat{A}^{-1} \vec{u})\). \label{item:sherman-morrison}

		            More generally, prove the \emph{Sherman-Morrison-Woodbury formula}: Let \(\mat{U}\) and \(\mat{V}\) be \(n \times k\) rectangular matrices, where \(k < n\) and \(\mat{A}\) is \(n \times n\). Then \(\mat{T} = \mat{I} + \mat{V}^\top \mat{A}^{-1} \mat{U}\) is nonsingular if and only if \(\mat{A} + \mat{U}\mat{V}^\top\) is nonsingular, in which case \((\mat{A} + \mat{U}\mat{V}^\top )^{-1} = \mat{A}^{-1} - \mat{A}^{-1} \mat{U}\mat{T}^{-1} \mat{V}^\top \mat{A}^{-1}\).
		            \begin{proof}
			            First, we have
			            \begin{align*}
				            (\mat{A} + \vec{u}\vec{v}^\top) \qty(\mat{A}^{-1} - \frac{\mat{A}^{-1}\vec{u}\vec{v}^\top\mat{A}^{-1}}{1 + \vec{v}^\top \mat{A}^{-1} \vec{u}}) & = \cancelto{\mat{I}}{\mat{A}\mat{A}^{-1}} - \frac{\cancel{\mat{A}\mat{A}^{-1}}\vec{u}\vec{v}^\top\mat{A}^{-1}}{1 + \vec{v}^\top \mat{A}^{-1} \vec{u}} + \vec{u}\vec{v}^\top\mat{A}^{-1} - \frac{\vec{u}\vec{v}^\top\mat{A}^{-1}\vec{u}\vec{v}^\top\mat{A}^{-1}}{1 + \vec{v}^\top \mat{A}^{-1} \vec{u}} \\
				                                                                                                                                                           & = \mat{I} - \frac{\vec{u}\vec{v}^\top\mat{A}^{-1} - \vec{u}\vec{v}^\top\mat{A}^{-1}(1 + \vec{v}^\top \mat{A}^{-1} \vec{u}) + \vec{u}\vec{v}^\top\mat{A}^{-1}\vec{u}\vec{v}^\top\mat{A}^{-1}}{1 + \vec{v}^\top \mat{A}^{-1} \vec{u}}                                                                    \\
				                                                                                                                                                           & = \mat{I} - \frac{\cancel{\vec{u}\vec{v}^\top\mat{A}^{-1}} - \cancel{\vec{u}\vec{v}^\top\mat{A}^{-1}} - \vec{u}\vec{v}^\top\mat{A}^{-1}\vec{v}^\top \mat{A}^{-1} \vec{u} + \vec{u}\vec{v}^\top\mat{A}^{-1}\vec{u}\vec{v}^\top\mat{A}^{-1}}{1 + \vec{v}^\top \mat{A}^{-1} \vec{u}}                      \\
				                                                                                                                                                           & = \mat{I} - \frac{\vec{v}^\top \mat{A}^{-1} \vec{u}(- \cancel{\vec{u}\vec{v}^\top\mat{A}^{-1}} + \cancel{\vec{u}\vec{v}^\top\mat{A}^{-1}})}{1 + \vec{v}^\top \mat{A}^{-1} \vec{u}}                                                                                                                     \\
				                                                                                                                                                           & = \mat{I},
			            \end{align*}
			            so \((\mat{A} + \vec{u}\vec{v}^\top)^{-1} = \mat{A}^{-1} - (\mat{A}^{-1}\vec{u}\vec{v}^\top\mat{A}^{-1})/(1 + \vec{v}^\top \mat{A}^{-1} \vec{u})\). Then, assume \(\mat{T} = \mat{I} + \mat{V}^\top \mat{A}^{-1} \mat{U}\) is nonsingular. We have
			            \begin{align*}
				            (\mat{A} + \mat{U}\mat{V}^\top )(\mat{A}^{-1} - \mat{A}^{-1} \mat{U}\mat{T}^{-1} \mat{V}^\top \mat{A}^{-1}) & = \cancelto{\mat{I}}{\mat{A}\mat{A}^{-1}} - \cancel{\mat{A}\mat{A}^{-1}} \mat{U}\mat{T}^{-1} \mat{V}^\top \mat{A}^{-1}                           \\
				                                                                                                                        & \phantom{{}=}+ \mat{U}\mat{V}^\top\mat{A}^{-1} - \mat{U}\mat{V}^\top\mat{A}^{-1} \mat{U}\mat{T}^{-1} \mat{V}^\top \mat{A}^{-1}                   \\
				                                                                                                                        & = \mat{I} - \mat{U}(\mat{T}^{-1} - \mat{I} + \underbrace{\mat{V}^\top\mat{A}^{-1}\mat{U}}_{\mat{T}-\mat{I}}\mat{T}^{-1})\mat{V}^\top\mat{A}^{-1} \\
				                                                                                                                        & = \mat{I} - \mat{U}(\cancel{\mat{T}^{-1}} - \cancel{\mat{I}} + \cancel{\mat{T}\mat{T}^{-1}} - \cancel{\mat{T}^{-1}})\mat{V}^\top\mat{A}^{-1}     \\
				                                                                                                                        & = \mat{I},
			            \end{align*}
			            so \(\mat{A} + \mat{U}\mat{V}^\top\) is nonsingular and \((\mat{A} + \mat{U}\mat{V}^\top )^{-1} = \mat{A}^{-1} - \mat{A}^{-1} \mat{U}\mat{T}^{-1} \mat{V}^\top \mat{A}^{-1}\). Finally, assume \(\mat{A} + \mat{U}\mat{V}^\top\) is nonsingular and \((\mat{A} + \mat{U}\mat{V}^\top )^{-1} = \mat{A}^{-1} - \mat{A}^{-1} \mat{U}\mat{B} \mat{V}^\top \mat{A}^{-1}\) for some \(\mat{B},\ k \times k\). Then we have
			            \begin{alignat*}{4}
				                     &  & \mat{I} & = (\mat{A} + \mat{U}\mat{V}^\top )(\mat{A} + \mat{U}\mat{V}^\top )^{-1}                                                                                                                                                       \\
				                     &  &         & = (\mat{A} + \mat{U}\mat{V}^\top)(\mat{A}^{-1} - \mat{A}^{-1} \mat{U}\mat{B} \mat{V}^\top \mat{A}^{-1})                                                                                                                       \\
				                     &  &         & = \cancelto{\mat{I}}{\mat{A}\mat{A}^{-1}} - \cancel{\mat{A}\mat{A}^{-1}}\mat{U}\mat{B} \mat{V}^\top \mat{A}^{-1} + \mat{U}\mat{V}^\top\mat{A}^{-1} - \mat{U}\mat{V}^\top\mat{A}^{-1} \mat{U}\mat{B} \mat{V}^\top \mat{A}^{-1} \\
				            \implies &  & \mat{0} & = \mat{U}(- \mat{B} + \mat{I} - \mat{V}^\top\mat{A}^{-1} \mat{U}\mat{B} )\mat{V}^\top \mat{A}^{-1}                                                                                                                            \\
				            \implies &  & \mat{0} & = \mat{I} - \underbrace{(\mat{I} + \mat{V}^\top\mat{A}^{-1} \mat{U})}_{\mat{T}}\mat{B}                                                                                                                                        \\
				            \implies &  & \mat{I} & = \mat{T}\mat{B},
			            \end{alignat*}
			            therefore \(\mat{T}\) is invertible and non-singular, and \(\mat{B} = \mat{T}^{-1}\).
		            \end{proof}

		      \item If you have a fast algorithm to solve \(\mat{A}\vec{x} = \vec{b}\), show how to build a fast solver for \(\mat{B}\vec{y} = \vec{c}\), where \(\mat{B} = \mat{A} + \vec{u}\vec{v}^\top\).
		            \begin{answer}
			            We have
			            \begin{align*}
				            \vec{y} & = \mat{B}^{-1} \vec{c}                                                                                                                                              \\
				                    & = (\mat{A} + \vec{u}\vec{v}^\top)^{-1} \vec{c}                                                                                                                      \\
				                    & \stackrel{\mathclap{\ref{item:sherman-morrison}}}{=} (\mat{A}^{-1} - (\mat{A}^{-1}\vec{u}\vec{v}^\top\mat{A}^{-1})/(1 + \vec{v}^\top \mat{A}^{-1} \vec{u})) \vec{c} \\
				                    & = \mat{A}^{-1}\vec{c} - (\mat{A}^{-1}\vec{u})\vec{v}^\top(\mat{A}^{-1}\vec{c})/(1 + \vec{v}^\top (\mat{A}^{-1} \vec{u})).
			            \end{align*}

			            Then, an algorithm to quickly solve this problem can be found in \cref{alg:quick-solve}.
			            \begin{algorithm}
				            \caption{An algorithm to quickly solve \((\mat{A} + \vec{u}\vec{v}^\top)\vec{y} = \vec{c}\) given an algorithm to quickly solve \(\mat{A}\vec{x} = \vec{b}\).}
				            \label{alg:quick-solve}
				            \KwData{\(\mat{A},\ \vec{c},\ \vec{u},\ \vec{v}\)}
				            \KwResult{\(\vec{y} = (\mat{A} + \vec{u}\vec{v}^\top)^{-1}\vec{c}\)}
				            Solve \(\mat{A}\vec{w} = \vec{c}\)\;
				            Solve \(\mat{A}\vec{x} = \vec{u}\)\;
				            \(\vec{y} \gets \vec{w} - \vec{x}\vec{v}^\top\vec{w}/(1 + \vec{v}^\top \vec{x})\)\;
			            \end{algorithm}

		            \end{answer}

	      \end{enumerate}

	      % 3
	\item Consider the linear system \(\mat{A}\vec{x} = \vec{b}\) where
	      \[
		      \mat{A} = \begin{bmatrix}
			      2 & 1 & 1 & 0 \\
			      4 & 3 & 3 & 1 \\
			      8 & 7 & 9 & 5 \\
			      6 & 7 & 9 & 8
		      \end{bmatrix}
		      \quad \text{and} \quad
		      \vec{b} = \begin{bmatrix}
			      4 \\ 11 \\ 29 \\ 30
		      \end{bmatrix}.
	      \]
	      \begin{enumerate}
		      \item Calculate the appropriate determinants to show that \(\mat{A}\) can be written as \(\mat{A} = \mat{L}\mat{U}\) where \(\mat{L}\) is a unit lower triangular matrix and \(\mat{U}\) is a non-singular upper triangular matrix. You may use MATLAB to compute the determinants. But the remaining parts of this problem should be done by paper-and-pencil.
		            \begin{answer}
			            \begin{align*}
				            \det\mat{A}_1 & = \mqty|1| = 1             \\
				            \det\mat{A}_2 & = \mqty|2      & 1         \\ 4 & 3| = 2 \\
				            \det\mat{A}_3 & = \mqty|2      & 1 & 1     \\ 4 & 3 & 3 \\ 8 & 7 & 9| = (2)(6)-(1)(12)+(1)(4) = 4 \\
				            \det\mat{A}_4 & = \mqty|2      & 1 & 1 & 0 \\ 4 & 3 & 3 & 1 \\ 8 & 7 & 9 & 5 \\ 6 & 7 & 9 & 8| = 2\mqty|3 & 3 & 1 \\ 7 & 9 & 5 \\ 7 & 9 & 8|-1\mqty|4 & 3 & 1 \\ 8 & 9 & 5 \\ 6 & 9 & 8| +1\mqty|4 & 3 & 1 \\ 8 & 7 & 5 \\ 6 & 7 & 8| + 0 = 8
			            \end{align*}
			            Since these determinants are all positive the LU decomposition without pivoting is possible.
		            \end{answer}

		      \item Compute the LU decomposition \(\mat{A} = \mat{L}\mat{U}\) and convert \(\mat{A}x = b\) into \(\mat{U}x = y\) where \(\mat{U}\) is upper triangular and solve for \(x\).
		            \begin{answer}
			            We have
			            \begin{align*}
				            \begin{bmatrix}
					            \alpha_{11}=2 & \vec{c}^{\top} = \mqty[1 & 1 & 0] \\
					            \displaystyle \vec{a} = \mqty[4                   \\8\\6] & \hat{\mat{A}} = \mqty[3&3&1\\7&9&5\\7&9&8]
				            \end{bmatrix}
			            \end{align*}
			            so \(\hat{l}_1 = \vec{a}/\alpha_{11} = \mqty[2\\4\\3]\), and
			            \begin{align*}
				            \hat{\mat{A}}-\hat{l}_1 \vec{c}^{\top} & = \mqty[3 & 3 & 1 \\7&9&5\\7&9&8]-\mqty[2\\4\\3]\mqty[1&1&0] \\
				                                                   & = \mqty[3 & 3 & 1 \\7&9&5\\7&9&8]-\mqty[2&2&0\\4&4&0\\3&3&0] \\
				                                                   & = \mqty[1 & 1 & 1 \\3&5&5\\4&6&8].
			            \end{align*}
			            Then the first step of LU gives
			            \begin{equation}
				            \mqty[1&0\\\hat{l}_1 &\mat{I}_{3}]\mqty[\alpha_{11} & \vec{c}^{\top}\\0 &\hat{\mat{A}}-\hat{l}_1\vec{c}^{\top}] = \mqty[1&0&0&0\\2&1&0&0\\4&0&1&0\\3&0&0&1] \mqty[2&1&1&0\\0&1&1&1\\0&3&5&5\\0&4&6&8] \label{eq:LU1}.
			            \end{equation}
			            Repeating this process on the lower right submatrix in $U_1$ from \cref{eq:LU1} we have
			            \begin{align*}
				            \begin{bmatrix}
					            \alpha_{22}=1 & \vec{c}^{\top} = \mqty[1 & 1] \\
					            \displaystyle \vec{a} = \mqty[3               \\4] & \hat{\mat{A}} = \mqty[5&5\\6&8]
				            \end{bmatrix}
			            \end{align*}
			            so \(\hat{l}_2 = \vec{a}/\alpha_{22} = \mqty[3\\4]\), and
			            \begin{align*}
				            \hat{\mat{A}}-\hat{l}_2 \vec{c}^{\top} & = \mqty[5 & 5 \\6&8]-\mqty[3\\4]\mqty[1&1] \\
				                                                   & = \mqty[5 & 5 \\6&8]-\mqty[3&3\\4&4] \\
				                                                   & = \mqty[2 & 2 \\2&4].
			            \end{align*}
			            Then the second step of LU gives
			            \begin{equation}
				            \mat{A} = \mqty[1&0&0&0\\2&1&0&0\\4&3&1&0\\3&4&0&1] \mqty[2&1&1&0\\0&1&1&1\\0&0&2&2\\0&0&2&4] \label{eq:LU2}.
			            \end{equation}
			            Repeating one final time on the lower \(2\times2\) submatrix in \cref{eq:LU2} yields
			            \begin{align*}
				            \begin{bmatrix}
					            \alpha_{33}=2                    & \vec{c}^{\top} = \mqty[2] \\
					            \displaystyle \vec{a} = \mqty[2] & \hat{\mat{A}} = \mqty[4]
				            \end{bmatrix}
			            \end{align*}
			            so \(\hat{l}_3 = \vec{a}/\alpha_{33} = \mqty[1]\), and
			            \begin{align*}
				            \hat{\mat{A}}-\hat{l}_3 \vec{c}^{\top} & = \mqty[4]-\mqty[1]\mqty[2] \\
				                                                   & = \mqty[2].
			            \end{align*}
			            So the third, and final, step of LU gives
			            \begin{equation}
				            \boxed{\mat{A} = \mqty[1&0&0&0\\2&1&0&0\\4&3&1&0\\3&4&1&1] \mqty[2&1&1&0\\0&1&1&1\\0&0&2&2\\0&0&0&2]} \label{eq:LU}.
			            \end{equation}
			            Now, if we have \(\mat{A}\vec{x}=\vec{b}\) then we have \(\mat{L}\mat{U}\vec{x}=\vec{b}\), where \(\mat{L}\) is invertible. So, \(\mat{U}\vec{x}=\mat{L}^{-1}\vec{b}=\vec{y}\), and we can find \(\vec{y}\) with
			            \begin{align*}
				            \vec{y}=\mat{L}^{-1}\vec{b} & = \mqty[1 & 0 & 0 & 0 \\-2 & 1 & 0 & 0\\2 & -3 & 1 & 0\\3 & -1 & -1 & 1]\mqty[4 \\ 11 \\ 29 \\ 30] \\
				                                        & =\mqty[4              \\-8+11\\8-33+29\\12-11-29+30] \\
				                                        & = \mqty[4             \\3\\4\\2].
			            \end{align*}
			            Then the transformed equation is
			            \[
				            \mat{U}\vec{x}=\vec{y} \implies \mqty[2&1&1&0\\0&1&1&1\\0&0&2&2\\0&0&0&2] \mqty[x_1\\x_2\\x_3\\x_4] = \mqty[4\\3\\4\\2].
			            \]
			            Starting at the bottom, clearly \(x_4 = 1\). Then we have \(2x_3+2x_4=4\implies 2x_3+2=4\implies x_3=1\). Then we have \(x_2+x_3+x_4=3\implies x_2+1+1=3\implies x_2=1\). Finally, we have \(2x_1+x_2+x_3=4\implies 2x_1 + 1+1 =4\implies x_1 = 1\).
			            So,
			            \[
				            \boxed{\vec{x} = \mqty[1\\1\\1\\1]}	.
			            \]
		            \end{answer}

		      \item Using \(\mat{L}\) and \(\mat{U}\) from the earlier steps, solve the new system \(\mat{A}\tilde{x} = \tilde{b}\) where
		            \[
			            \tilde{b} =
			            \begin{bmatrix}
				            5 \\ 15 \\ 41 \\ 45
			            \end{bmatrix}.
		            \]
		            Do NOT perform the Gaussian elimination from scratch. You already have \(\mat{U}\) and \(\mat{L}\).
		            \begin{answer}
			            So, \(\mat{U}\tilde{x}=\mat{L}^{-1}\tilde{b}=\tilde{y}\), and we can find \(\tilde{y}\) with
			            \begin{align*}
				            \tilde{y}=\mat{L}^{-1}\tilde{b} & = \mqty[1 & 0 & 0 & 0 \\-2 & 1 & 0 & 0\\2 & -3 & 1 & 0\\3 & -1 & -1 & 1]\mqty[5 \\ 15 \\ 41 \\ 45] \\
				                                            & =\mqty[5              \\-10+15\\10-45+41\\15-15-41+45] \\
				                                            & = \mqty[5             \\5\\6\\4].
			            \end{align*}
			            Then the transformed equation is
			            \[
				            \mat{U}\tilde{x}=\tilde{y} \implies \mqty[2&1&1&0\\0&1&1&1\\0&0&2&2\\0&0&0&2] \mqty[\tilde{x}_1\\\tilde{x}_2\\\tilde{x}_3\\\tilde{x}_4] = \mqty[5\\5\\6\\4].
			            \]
			            Starting at the bottom, clearly \(\tilde{x}_4 = 2\). Then we have \(2\tilde{x}_3+2\tilde{x}_4=6\implies 2\tilde{x}_3+4=6\implies \tilde{x}_3=1\). Then we have \(\tilde{x}_2+\tilde{x}_3+\tilde{x}_4=5\implies \tilde{x}_2+1+2=5\implies \tilde{x}_2=2\). Finally, we have \(2\tilde{x}_1+\tilde{x}_2+\tilde{x}_3=5\implies 2\tilde{x}_1 + 2 + 1 = 5\implies \tilde{x}_1 = 1\). So,
			            \[
				            \boxed{\tilde{x} = \mqty[1\\2\\1\\2]}	.
			            \]
		            \end{answer}
	      \end{enumerate}

	      % 4
	\item Let \(\mat{A}\) have the following economy (thin) SVD:
	      \[
		      \mat{A} = \begin{bmatrix}
			      1 & 0    \\
			      0 & 0.6  \\
			      0 & -0.8
		      \end{bmatrix}
		      \begin{bmatrix}
			      4 & 0 \\
			      0 & 3
		      \end{bmatrix}
		      \begin{bmatrix}
			      0.6  & -0.8 \\
			      -0.8 & -0.6 \\
		      \end{bmatrix}^\top.
	      \]
	      Answer the following questions without forming \(\mat{A}\) and without using Matlab. None of these questions require using numerical software. At most, you need to be able to perform simple matrix arithmetic using paper-and-pencil.
	      \begin{enumerate}
		      \item Find \(\rank(\mat{A}),\ \|\mat{A}\|_2\) and \(\|\mat{A}\|_F\).
		            \begin{answer}
			            We have
			            \begin{align*}
				            \rank(\mat{A}) & = 2,                                                                                                   \\
				            \|\mat{A}\|_2  & = \max_i \sqrt{\lambda_i(A^\top A)}                                                                    \\
				                           & = \sqrt{\sigma_1^2(\mat{A})}                                                                           \\
				                           & = |\sigma_1(\mat{A})| = 4,                                                                             \\
				            \|\mat{A}\|_F  & = \sqrt{\trace((\mat{U}\mat{\Sigma}\mat{V}^\top)^\top(\mat{U}\mat{\Sigma}\mat{V}^\top))}               \\
				                           & = \sqrt{\trace(\cancel{\mat{V}^\top\mat{V}}\mat{\Sigma}^\top\cancel{\mat{U}^\top\mat{U}}\mat{\Sigma})} \\
				                           & = \sqrt{\trace^2(\mat{\Sigma})}                                                                        \\
				                           & = |\trace(\mat{\Sigma})|                                                                               \\
				                           & = 4 + 3 = 7.
			            \end{align*}
		            \end{answer}

		      \item Find an orthonormal basis for \(\range(\mat{A}),\ \nullsp(\mat{A}),\ \range(\mat{A}^\top)\), and \(\nullsp(\mat{A}^\top)\).
		            \begin{answer}
			            We have
			            \begin{alignat*}{4}
				            \range(\mat{A})       & = \spn \qty{\begin{bmatrix}1\\ 0\\ 0\end{bmatrix}, \begin{bmatrix} 0\\ 0.6\\ -0.8\end{bmatrix}},                   \\
				            \nullsp(\mat{A})      & = \spn \emptyset,                                                                                                  \\
				            \range(\mat{A}^\top)  & = \spn \qty{\begin{bmatrix}0.6\\ -0.8\end{bmatrix}, \begin{bmatrix}-0.8\\ -0.6\end{bmatrix}},                      \\
				            \nullsp(\mat{A}^\top) & = \reals^3 \setminus \spn \qty{\begin{bmatrix}1\\ 0\\ 0\end{bmatrix}, \begin{bmatrix} 0\\ 0.6\\ -0.8\end{bmatrix}} \\
				                                  & = \spn \qty{\begin{bmatrix} 0\\ 0.8\\ 0.6\end{bmatrix}}.
			            \end{alignat*}
			            Note that the given sets consist of orthogonal unit vectors, so they are orthonormal bases for the sets which they span.
		            \end{answer}

		      \item Form the optimal rank-1 approximation \(\mat{A}_1\) to \(\mat{A}\) in the 2-norm? What is the error matrix \(\mat{A} - \mat{A}_1\), and what is the norm of the error \(\|\mat{A} - \mat{A}_1\|_2\)? The error matrix and the norm of the error follow directly from the SVD; no computation is needed.
		            \begin{answer}
			            We have
			            \begin{align*}
				            \mat{A}_1                 & = \begin{bmatrix}
					                                          1 & 0    \\
					                                          0 & 0.6  \\
					                                          0 & -0.8
				                                          \end{bmatrix}
				            \begin{bmatrix}
					            4 & 0 \\
					            0 & 0
				            \end{bmatrix}
				            \begin{bmatrix}
					            0.6  & -0.8 \\
					            -0.8 & -0.6 \\
				            \end{bmatrix}^\top                                                                                \\
				                                      & = \begin{bmatrix}
					                                          1 \\
					                                          0 \\
					                                          0
				                                          \end{bmatrix}
				            \begin{bmatrix}
					            4
				            \end{bmatrix}
				            \begin{bmatrix}
					            0.6  \\
					            -0.8 \\
				            \end{bmatrix}^\top,                                                                               \\
				            \mat{A} - \mat{A}_1       & = \begin{bmatrix}
					                                          1 & 0    \\
					                                          0 & 0.6  \\
					                                          0 & -0.8
				                                          \end{bmatrix} \qty(\begin{bmatrix}
						                                                             4 & 0 \\
						                                                             0 & 3
					                                                             \end{bmatrix} - \begin{bmatrix}
						                                                                             4 & 0 \\
						                                                                             0 & 0
					                                                                             \end{bmatrix})\begin{bmatrix}
					                                                                                           0.6  & -0.8 \\
					                                                                                           -0.8 & -0.6 \\
				                                                                                           \end{bmatrix}^\top \\
				                                      & = \begin{bmatrix}
					                                          1 & 0    \\
					                                          0 & 0.6  \\
					                                          0 & -0.8
				                                          \end{bmatrix}\begin{bmatrix}
					                                                       0 & 0 \\
					                                                       0 & 3
				                                                       \end{bmatrix}\begin{bmatrix}
					                                                                    0.6  & -0.8 \\
					                                                                    -0.8 & -0.6 \\
				                                                                    \end{bmatrix}^\top,                       \\
				            \|\mat{A} - \mat{A}_1\|_2 & = 3.
			            \end{align*}
		            \end{answer}
	      \end{enumerate}

	      % 5
	\item (From Golub and van Loan) Given \(\mat{A} \in \reals^{m \times n}\), let \(\sigma_1\) denote the largest singular value of \(\mat{A}\). Prove that
	      \[
		      \sigma_1(\mat{A}) = \max_{\substack{\vec{y} \in \reals^m, \\ \vec{x} \in \reals^n}} \frac{\vec{y}^\top \mat{A} \vec{x}}{\|\vec{y}\|_2\|\vec{x}\|_2}.
	      \]
	      \begin{proof}
		      For \(\vec{y},\vec{x}\neq 0\) we have
		      \begin{align}
			      \max_{\substack{\vec{y} \in \reals^m,   \\ \vec{x} \in \reals^n}} \frac{\vec{y}^\top \mat{A} \vec{x}}{\|\vec{y}\|_2\|\vec{x}\|_2} &= \max_{\substack{\vec{y} \in \reals^m, \\ \vec{x} \in \reals^n}} \qty(\frac{\vec{y}}{\norm{\vec{y}}_2})^{\top} \frac{\mat{A}\vec{x}}{\norm{\vec{x}}_2} \nonumber \\
			       & = \max_{\substack{\vec{y}\in\reals^n \\\norm{\vec{y}}=1, \\ \vec{x} \in \reals^n}} \vec{y}^{\top} \frac{\mat{A}\vec{x}}{\norm{\vec{x}}_2} \label{eq:Golubsetup}
		      \end{align}
		      which is the inner product, or the projection of a unit vector \(\vec{y}\) onto \(\frac{\mat{A}\vec{x}}{\norm{\vec{x}}_2}\). The maximum of this quantity is achieved when \(\vec{y}\) is the unit vector in the direction of \(\frac{\mat{A}\vec{x}}{\norm{\vec{x}}_2}\). In other words,
		      \begin{align*}
			      \vec{y} & = \frac{\displaystyle\frac{\mat{A}\vec{x}}{\norm{\vec{x}}}_2}{\norm{\displaystyle\frac{\mat{A}\vec{x}}{\norm{\vec{x}}_2}}_2} \\
			              & = \frac{\mat{A}\vec{x}}{\norm{\mat{A}\vec{x}}_2}.
		      \end{align*}
		      Then in \cref{eq:Golubsetup} we have
		      \begin{align*}
			      \max_{\substack{\vec{y} \in \reals^m,                                                                                                 \\ \vec{x} \in \reals^n}} \frac{\vec{y}^\top \mat{A} \vec{x}}{\|\vec{y}\|_2\|\vec{x}\|_2} &= \max_{\substack{\vec{y}\in\reals^n \\\norm{\vec{y}}=1, \\ \vec{x} \in \reals^n}} \vec{y}^{\top} \frac{\mat{A}\vec{x}}{\norm{\vec{x}}_2}  \\
			       & = \max_{\vec{x} \in \reals^n} \qty( \frac{\mat{A}\vec{x}}{\norm{\mat{A}\vec{x}}_2})^{\top} \frac{\mat{A}\vec{x}}{\norm{\vec{x}}_2} \\
			       & = \max_{\vec{x} \in \reals^n} \frac{\norm{\mat{A}\vec{x}}_2^2}{\norm{\mat{A}\vec{x}}_2\norm{\vec{x}}_2}                            \\
			       & = \max_{\vec{x} \in \reals^n} \frac{\norm{\mat{A}\vec{x}}_2}{\norm{\vec{x}}_2}                                                     \\
			       & = \norm{\mat{A}}_2                                                                                                                 \\
			       & = \sigma_1(\mat{A}).
		      \end{align*}
	      \end{proof}

	      % 6
	\item Let \(\mat{A} \in \reals^{m \times n}\) have the singular value decomposition
	      \[
		      \mat{A} = \sum_{j = 1}^r \sigma_j \vec{u}_j\vec{v}_j^\top.
	      \]
	      We showed in class that the matrix
	      \[
		      \mat{A}_k = \sum_{j = 1}^k \sigma_j \vec{u}_j\vec{v}_j^\top
	      \]
	      is an optimal rank-\(k\) approximation to \(\mat{A}\) in the 2-norm and \(\|\mat{A} - \mat{A}_k\| = \sigma_{k+1}\). But this minimizer (optimal approximant) is not unique. In other words, using truncated SVD we can find another rank-\(k\) matrix \(\tilde{\mat{A}}_k \in \reals^{m\times n}\) such that \(\|\mat{A} - \tilde{\mat{A}}_k\| = \sigma_{k+1}\). In this problem, you will prove this fact.

	      Define
	      \begin{equation}
		      \tilde{\mat{A}}_k = \sum_{j = 1}^k(\sigma_j - \eta_j)\vec{u}_j\vec{v}_j^\top \quad \text{where} \quad 0 \leq |\eta_j| < \sigma_{k + 1}. \label{eq:a-tilde}
	      \end{equation}
	      Show that has \(\tilde{\mat{A}}_k\) in \eqref{eq:a-tilde} has rank-\(k\) and minimizes the error, i.e, \(\|\mat{A} - \tilde{\mat{A}}_k\| = \sigma_{k+1}\).
	      \begin{proof}
		      Observe that \(\sum_{j = 1}^k(\sigma_j - \eta_j)\vec{u}_j\vec{v}_j^\top\) is the SVD of \(\tilde{\mat{A}}_k\), with singular values of \((\sigma_j - \eta_j)\). Since \(|\eta_j| < \sigma_{k + 1} \leq \sigma_j\) for all \(1 \leq j \leq k\), then \(\sigma_j - \eta_j > 0\), and \(\tilde{\mat{A}}_k\) has \(k\) non-zero singular values, so it has rank \(k\). As well, we have
		      \begin{align*}
			      \mat{A} - \tilde{\mat{A}}_k & = \sum_{j = 1}^r \sigma_j \vec{u}_j\vec{v}_j^\top - \sum_{j = 1}^k(\sigma_j - \eta_j)\vec{u}_j\vec{v}_j^\top                                             \\
			                                  & = \sum_{j = 1}^k \qty[\sigma_j \vec{u}_j\vec{v}_j^\top - (\sigma_j - \eta_j)\vec{u}_j\vec{v}_j^\top] + \sum_{j = k+1}^r \sigma_j \vec{u}_j\vec{v}_j^\top \\
			                                  & = \sum_{j = 1}^k \eta_j\vec{u}_j\vec{v}_j^\top + \sum_{j = k+1}^r \sigma_j \vec{u}_j\vec{v}_j^\top.
		      \end{align*}
		      Note, then, that this is the SVD of \(\mat{A} - \tilde{\mat{A}}_k\), with eigenvalues of \(|\eta_j|\) for \(1 \leq j \leq k\) and \(\sigma_j\) for \(k+1 \leq j \leq r\). Since \(|\eta_j| < \sigma_{k+1}\) for all \(j\) and \(\sigma_{k+1} \geq \sigma_{k}\) for all \(j\), we must have
		      \[
			      \|\mat{A} - \tilde{\mat{A}}_k\|_2 = \sigma_{k+1},
		      \]
		      as required.
	      \end{proof}

	      % 7
	\item Approximation of a Fingerprint Image via SVD: You will complete the Matlab Script \texttt{Homework3Problem7.m} to answer this question. You will attach the completed script to the .pdf file.

	      Before start completing your code, it will help use the commands \texttt{help svd}, \texttt{help subplot}, \texttt{help semilogy} to understand how to use these functions.

	      Lines 7-10 load the image into Matlab and plot it. The resulting matrix \(\mat{A}\) in your workspace is a \(1133 \times 784\) matrix with entries consisting of only ones and zeroes; ones correspond to the white spots in the image and zeroes to the black spots.
	      \begin{enumerate}
		      \item Compute the short (reduced) SVD of \(\mat{A}\) using Matlab. Then plot the normalized singular values under \texttt{subplot(2,1,2)}. This figure might be hard to read due to the scale of the \(y\)-axis. Then plot only the leading 700 normalized singular values under \texttt{subplot(2,1,2)}. What do you observe? What would be your decision for the (numerical) rank of \(\mat{A}\) based on these comments? Justify your reasoning. Does the result of the built-in rank command coincide with your decision?
		            \begin{answer}
			            Plots of the singular values of \(\mat{A}\) can be found in \cref{fig:singvals}. We notice a significant drop off in singular values after the \(654^\text{th}\) singular value. Because of this, the numerical rank of \(\mat{A}\) should be 654. Indeed, this is the computed rank of \(\mat{A}\).
			            \begin{figure}[H]
				            \centering\begin{tikzpicture}
					            \begin{groupplot}[
							            width = .9\textwidth,
							            height = .4\textwidth,
							            xmax = 800,
							            group style = {group size = 1 by 2},
							            ymode = log,
							            extra x ticks = {654},
							            restrict y to domain=-167:,
							            log basis y=10,
							            ylabel=\(\sigma_i\),
							            xlabel=\(i\)
						            ]
						            \nextgroupplot[ymin={1e-167},ymax=1e13]
						            \addplot table [
								            y expr = \thisrowno{0},
								            x expr = \coordindex +1
							            ] {out/all_singular_values7.dat};
						            \addplot[dashed] coordinates {(654, 1e-167)  (654, 1e13)};
						            \nextgroupplot[ymin={1e-18},ymax=1e2]
						            \addplot table [
								            y expr = \thisrowno{0},
								            x expr = \coordindex +1,
								            restrict x to domain=1:700
							            ] {out/all_singular_values7.dat};
						            \addplot[dashed] coordinates {(654, 1e-18)  (654, 1e2)};
					            \end{groupplot}
				            \end{tikzpicture}
				            \caption{Plots showing all singular values of \(\mat{A}\) and just the first 700 singular values of \(\mat{A}\). Note the large gap between values that occurs at the calculated rank of \(\mat{A}\) (654).}\label{fig:singvals}
			            \end{figure}
		            \end{answer}

		      \item Construct the optimal rank-\(k\) approximation \(\mat{A}_k\) to \(\mat{A}\) in the 2-norm for \(k = 1, k = 10\), and \(k = 50\). In each case, compute the relative error \(\frac{\|\mat{A} - \mat{A}_k\|_2}{\|\mat{A}\|_2}\). Note that you do NOT need to form \(\mat{A} - \mat{A}_k\) to compute these error values; the singular values are all you need. Use \texttt{subplot(2,2,1)} to plot the original image in the top left corner. Then, use the \texttt{imshow} command on the three low-rank approximations and plot them in the \texttt{subplot(2,2,2),} \texttt{subplot(2,2,3),} and \texttt{subplot(2,2,4)} spots. Put appropriate titles on each plot, e.g., ``original image'', ``rank-1 approximation'' etc., using the title command. Plots without appropriate labels and explanations will lose points. These plots need to be attached to the .pdf file.
		            \begin{answer} The plots of the rank-\(k\) approximations, for \(k=1,10,50\), can be found in \cref{fig:Thums}, with program output displayed below:
			            \begin{mdframed}[backgroundcolor=lightgray]
				            \inputminted{text}{out/output.txt}
			            \end{mdframed}
			            \begin{figure}[H]
				            \centering
				            \begin{subfigure}{0.225\textwidth}
					            \centering
					            \includegraphics[width=\textwidth]{out/Original matrix as image check.png}
					            \caption{Original Image}\label{fig:ogThumb}
				            \end{subfigure}\qquad
				            \begin{subfigure}{0.225\textwidth}
					            \centering
					            \includegraphics[width=\textwidth]{out/Rank_1_Approximation.png}
					            \caption{\(k=1\)}\label{fig:Thumb1}
				            \end{subfigure}\\
				            \begin{subfigure}{0.225\textwidth}
					            \centering
					            \includegraphics[width=\textwidth]{out/Rank_10_Approximation.png}
					            \caption{\(k=10\)}\label{fig:Thumb10}
				            \end{subfigure}\qquad
				            \begin{subfigure}{0.225\textwidth}
					            \centering
					            \includegraphics[width=\textwidth]{out/Rank_50_Approximation.png}
					            \caption{\(k=50\)}\label{fig:Thumb50}
				            \end{subfigure}
				            \caption{Original fingerprint image compared to its optimal rank-\(k \) approximations.}\label{fig:Thums}
			            \end{figure}
			            The code for this question is given below:
			            \begin{mdframed}[backgroundcolor=lightgray]
				            \rustcode[fontsize=\footnotesize]{src/prob7.rs}
			            \end{mdframed}
		            \end{answer}
	      \end{enumerate}

	      % 8
	\item In the previous problem you used SVD to compress a black-and-white image. In this example, you will use SVD to compute a low-rank approximation to the color image hokiebirdwithacat.jpg. You will complete the Matlab Script Homework3Problem8.m to answer this question. You will attach the completed script to the .pdf file.

	      Lines 7-20 load the image into Matlab, plot the original image, and extract the images correspond to every color layer, and convert these three layer images to double precision matrices A1, A2, and A3.
	      \begin{enumerate}
		      \item Compute the SVDs of every layer A1, A2, and A3. Then compute the vector of normalized singular values for every layer. Using the subplot comment (and logarithmic y-axis), plot all three vectors in Figure 2.
		            \begin{answer}
			            The normalized singular values for each channel can be found in \cref{fig:birdsingvals}. Note that the color of the plot indicates which channel the singular values are from.
			            \begin{figure}[H]
				            \centering\begin{tikzpicture}
					            \begin{groupplot}[
							            width = .9\textwidth,
							            height = .3\textwidth,
							            xmin = -1,
							            xmax = 530,
							            group style = {group size = 1 by 3},
							            ymode = log,
							            ylabel=\(\sigma_i\),
							            xlabel=\(i\)
						            ]
						            \nextgroupplot
						            \addplot+[red, mark options={fill=red}] table [
								            y = A1,
								            x expr = \coordindex +1
							            ] {out/all_singular_values8.dat};
						            \nextgroupplot
						            \addplot+[green, mark options={fill=green}] table [
								            y = A2,
								            x expr = \coordindex +1
							            ] {out/all_singular_values8.dat};
						            \nextgroupplot
						            \addplot+[blue, mark options={fill=blue}] table [
								            y = A3,
								            x expr = \coordindex +1
							            ] {out/all_singular_values8.dat};
					            \end{groupplot}
				            \end{tikzpicture}
				            \caption{Plots showing all normalized singular values of hokiebirdwithacat, split into red, green, and blue components. The color of the plot indicates which component the singular values were calculated from}
				            \label{fig:birdsingvals}
			            \end{figure}
		            \end{answer}

		      \item \underline{Use the same error tolerance for every layer}: Find the smallest rank-k for each layer such that the relative error in each layer is less than

		            \begin{tasks}[label = {(\roman*)}, label-width = 1.75em, label-offset = 0em](5)
			            \task*(1)[]
			            \task 10\%
			            \task 5\%
			            \task 1\%
			            \task*(1)[]
		            \end{tasks}

		            So, you will have three approximations. For each case, plot the low-rank image (either as a new figure or all in one plot using subplot). These plots need to be attached to the .pdf file. Make sure to clearly indicate the rank of ever layer and which plot corresponds to which error tolerance. Did you obtain the same k value for every layer? Comment on your results.
		            \begin{answer}
			            The approximations can be found in \cref{fig:equal-tol}. Note that different \(k\) values are chosen per channel, and that the red channel always uses the smallest \(k\), while blue uses the largest.
			            \begin{figure}[H]
				            \centering
				            \begin{subfigure}{0.225\textwidth}
					            \centering
					            \includegraphics[width=\textwidth]{out/RGB Image Conversion Check.png}
					            \caption{Original Image}
				            \end{subfigure}\hfill
				            \begin{subfigure}{0.225\textwidth}
					            \centering
					            \includegraphics[width=\textwidth]{out/err_all_0.1_Approximation.png}
					            \caption{\(tol = 10\%,\ k = 6, 7, 10\)}
				            \end{subfigure}\hfill
				            \begin{subfigure}{0.225\textwidth}
					            \centering
					            \includegraphics[width=\textwidth]{out/err_all_0.05_Approximation.png}
					            \caption{\(tol = 5\%,\ k = 12, 17, 23\)}
				            \end{subfigure}\hfill
				            \begin{subfigure}{0.225\textwidth}
					            \centering
					            \includegraphics[width=\textwidth]{out/err_all_0.01_Approximation.png}
					            \caption{\(tol = 1\%,\ k = 66, 89, 119\)}
					            \label{fig:1-percent-tol}
				            \end{subfigure}
				            \caption{Original hokie bird image compared to its optimal approximations for different error tolerances. Selected \(k\) values for the given tolerances are shown.}\label{fig:equal-tol}
			            \end{figure}
		            \end{answer}

		      \item \underline{Use different error tolerances for different layers}: Now we will choose different tolerances for different values. Pick three different selections:
		            \begin{itemize}
			            \item 50\% error for Layer 1, 1\% error for Layer 2, and 1\% error for Layer 3,
			            \item 1\% error for Layer 1, 50\% error for Layer 2, and 1\% error for Layer 3,
			            \item 1\% error for Layer 1, 1\% error for Layer 2, and 50\% error for Layer 3,
		            \end{itemize}
		            So, you will have three approximations. For each case, plot the low-rank image (either as a new figure or all in one plot using subplot). These plots need to be attached to the .pdf file. Make sure to clearly indicate the rank of ever layer and which plot corresponds to which error tolerance. Comment on your results.
		            \begin{answer}
			            The approximations can be found in \cref{fig:channel-tol}. Note that the chosen \(k\) values are only for the channel with the 50\% error tolerance - since the \(k\) values for 1\% error tolerance in each channel are given in \cref{fig:1-percent-tol} above. The 50\% error in blue looks much clearer than the other two channels - this is probably due to the relative lack of blue in the original image compared to the other two channels. Interestingly, the chosen \(k = 1\) for all channels - since the rank 1 approximation of each channel includes more than 50\% of the original information.
			            \begin{figure}[H]
				            \centering
				            \begin{subfigure}{0.225\textwidth}
					            \centering
					            \includegraphics[width=\textwidth]{out/50-err-red.png}
					            \caption{Red \(k = 1\)}
				            \end{subfigure}\qquad
				            \begin{subfigure}{0.225\textwidth}
					            \centering
					            \includegraphics[width=\textwidth]{out/50-err-green.png}
					            \caption{Green \(k = 1\)}
				            \end{subfigure}\qquad
				            \begin{subfigure}{0.225\textwidth}
					            \centering
					            \includegraphics[width=\textwidth]{out/50-err-blue.png}
					            \caption{Blue \(k = 1\)}
				            \end{subfigure}
				            \caption{Optimal approximations with a single channel constrained to 50\% error tolerance, with the rest of the channels at 1\%. The channel with 50\% tolerance is labeled, in addition to its chosen \(k\) value.}\label{fig:channel-tol}
			            \end{figure}

			            The code for this question is given below:
			            \begin{mdframed}[backgroundcolor=lightgray]
				            \rustcode[fontsize=\footnotesize]{src/prob8.rs}
			            \end{mdframed}

			            As well, additional output information, such as the actual computed relative errors of approximations, as well as verification of the correctly chosen \(k\) values, can be found below:
			            \begin{mdframed}[backgroundcolor=lightgray]
				            \inputminted{text}{out/prob8-output.txt}
			            \end{mdframed}
		            \end{answer}
	      \end{enumerate}
\end{enumerate}

\end{document}